\section*{Lambda Calculus}

\begin{itemize}
	\item Lambda calculus has variables, functions and function application. Instead of \texttt{(fun x -> e)} we write: $\lambda x. e$
	
	\item The only values are closed functions
	
	\item To substitute value $v$ for variable $x$ in expression $e$ we replace all free occurrences of $x$ in $e$ by $v$: $e\{v / x\}$
	
	\item Function application is  interpreted by substitution.
	
	\item In \texttt{fun y -> x + y}, \texttt{x} is said to be free and \texttt{y} is bound by \texttt{fun y}.
	\item A term without free variables is closed, else it is open.
	
	\item Two terms that differ only by consistent renaming of bound variables are alpha equivalent.
	\item To avoid accidently capturing a free variable by a substitution $e_1 \{e_2 / x\}$, we first pick an alpha equivalent version of $e_1$ such that the bound variables do not mention the free variables of $e_2$.
	
	\item Operational Semantics is a way to give meaning to a program (interpreter) using inference rules. $exp \Downarrow v$ means $exp$ evaluates to $v$.

	\item With inference rules we can build up derivation or proof trees. Leaves of the tree are axioms.
\end{itemize}
