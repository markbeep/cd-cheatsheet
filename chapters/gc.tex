\section*{Garbage Collection}
\begin{itemize}
	\item Garbage: An object \texttt{x} is reachable iff a register contains a pointer to \texttt{x} or another reachable object \texttt{y} contains a pointer to \texttt{x}.
	
	\item Reachable objects can be found by starting from registers and following all pointers

	\item Mark and Sweep
	
	When memory runs out, GC executes two phases: mark phase: trace reachable objects; sweep phase: collects garbage objects (extra bit reserved for memory management)
	
	Pointer reversal can be used to allow auxiliary data to be stored in the objects.
	
	\item Stop and Copy
	
	Memory is organized into two areas: Old space (used for allocation), new space (use as a reserve for GC)
	
	When old space is full all reachable objects are moved, old and new are swapped. The new space is partitioned into copied and scanned, copied and empty regions.
	
	Copied and scanned contain objects whose pointer fields were followed and fixed, while the copied region contains the objects that where copied but whose pointer filed were not (yet) fixed.
	
	\item Reference Counting
	
	Store number of references in the object itself, assignments modify that number. If the reference count is zero, remove the object. Cannot collect circular structures.
	
\end{itemize}